\chapter*{Abstract}
The need for comprehensive surveillance and detection of viral diseases in livestock has led to the use of \ac{NGS} data analysis pipelines for viruses such as \ac{AIV}, poxviruses and \ac{FMDV}. To achieve accurate analysis of isolates at the molecular level, we developed three workflows on the Galaxy platform that use different approaches of reference-based mapping with a dynamically generated reference sequence to enable rapid and informative genetic monitoring of viral origins, relationships and structures. The pipelines are based on components of globally deployed SARS-CoV-2 Galaxy workflows with Illumina-sequenced input data and are characterised by avoiding computationally intensive \textit{de novo} assembly and instead integrating reference sequence selection into the pipelines. We show that mapping-based pipelines can generate full-length consensus genomes useful for downstream tasks such as phylogenetic context analysis and mutation detection without requiring the user to have domain-specific knowledge for the selection of qualified reference. While for short viral genomes like FMDV we use a split method that integrates assembly in the reference selection process, we show that a fix reference for Capripoxviruses is sufficient for high quality mapping and consensus sequence construction. By providing ready-to-use Galaxy workflows that allow professionals in the field to perform viral genome analyses for poxviruses, AIV and FMDV we can expand our knowledge of disease outbreaks and ultimately deepen our understanding of viral genomes from animal isolates.