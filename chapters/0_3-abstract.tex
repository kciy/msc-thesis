\chapter*{Abstract}
The need for comprehensive monitoring and detection of viral diseases in livestock has led to the use of \ac{NGS} data analysis for viruses such as \ac{AIV}, poxviruses, and \ac{FMDV}. To achieve sensitive analysis of samples on molecular level, genomic analysis pipelines have been developed on the Galaxy platform. These pipelines utilise a reference-based mapping approach and dynamically searched reference sequences to enable rapid and insightful genetic surveillance of viral origins, relationships and structures. They are based on globally used SARS-CoV-2 Galaxy workflows with \ac{NGS} input data and are characterised by avoiding computationally expensive \textit{de novo} assembly, instead integrating the selection of reference sequence into the pipelines. We demonstrate that mapping-based pipelines can generate full-length consensus genomes that are useful for downstream tasks such as phylogenetic context analysis and mutation detection, without requiring domain-specific knowledge from the user. By providing ready-to-use Galaxy workflows and thereby entitling field professionals to run genomic analysis, we can expand our knowledge regarding viral disease outbreaks and ultimately deepen the understanding of viral genomes from viral livestock isolates.
