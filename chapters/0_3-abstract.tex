\chapter*{Abstract}
The need for comprehensive surveillance and detection of viral diseases in livestock has led to the use of \ac{NGS} data analysis pipelines for viruses such as \ac{AIV}, poxviruses and \ac{FMDV}. To achieve accurate analysis of samples at the molecular level, we developed genomic three workflows on the Galaxy platform that use a reference-based mapping approach with a dynamically generated reference sequence to enable rapid and informative genetic monitoring of viral origins, relationships and structures. The pipelines are based on globally deployed SARS-CoV-2 Galaxy workflows with \ac{NGS} input data and are characterised by avoiding computationally intensive \textit{de novo} assembly and instead integrating reference sequence selection into the pipelines. We show that mapping-based pipelines can generate full-length consensus genomes useful for downstream tasks such as phylogenetic context analysis and mutation detection without requiring the user to have domain-specific knowledge. By providing ready-to-use Galaxy workflows that allow professionals in the field to perform viral genome analyses, we can expand our knowledge of viral disease outbreaks and ultimately deepen our understanding of viral genomes from viral animal isolates.