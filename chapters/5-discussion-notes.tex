\begin{itemize}
    \item reference-based mapping pipelines for surveillance
    \item ready-to-use, require user interaction at times to pick suitable reference sequence
    \item transfer of knowledge from SARS-CoV-2 pipelines to other viruses
    \item on Galaxy, poxvirus WF on IWC for testing and versioning
\end{itemize}

\subsubsection*{Poxvirus Analysis}
\ac{LSDV} interesting for all poxviruses and adjustable due to \acp{ITR} and tiled-amplicon approach \\
further pox viruses, pipelines can be more or less easily applied/adjusted 


\subsubsection*{Avian Influenza Virus Analysis}

AIV: Choice of reference (issue of reference ambiguity), why not BLAST or lookup in GISAID with many hundreds of thousands of RNA virus genomes; \\
AIV: Alternatives, such as read classification by mapping to a large database of influenza sequences and subsequent de novo assembly can help to resolve this issue, such pipelines are often complex, slow, and require expertise that is not necessarily available in routine
surveillance or public health laboratories. S

However using an optimal reference is ideal, since for later advanced applications, such as transmission events, or study of intra-host variation, the closest possible reference may be necessary. Whilst BLAST performed well for individual read classification, it is often too slow for general application. With regards to de novo assembly, in the cases where not enough initial data exists to assemble fragments, mapping allows analysis of limited fragments. 

Make phylogenetic trees publicly accessible, not one sample per strain but in high resolution and greater details, strains from different countries,


\subsubsection*{Foot-and-mouth Disease Virus Analysis}
If the goal in a broad and rapid surveillance is a high number of sample throughput and analysis, assembly is too cost and time sensitive for large viral genomes. However small viral genomes such as FMDV perform reasonably faster in a de novo assembly than larger DNA genomes and achieve sufficiently good results. Long assembled contigs can be used for choosing a good, representative reference sequence that can be used for mapping. BLAST provides a large database however needs regular updates to deliver current hits and proper surveillance, also relevant for downstream/phylogeny.
Workflow in two parts requires knowledge and interaction by the user -> error prone?

Nevertheless this is the most suitable approach for high resolution genomic analysis, as a direct mapping fails due to not fitting reference sequences

--------------\\
Workflows that solve common problems, provide useful information, are user-friendly, customisable, extendable

limitations \\

``The high sensitivity of the \ac{NGS} technology ensures that major kinds of viral pathogens in mixed samples can be detected.''
One strength of \ac{NGS} is that it can be used to detect emerging viral diseases with a high genetic variation. Like \ac{AIV}. Since it can analyse a full sequence instead of targeting a specific gene. -> makes sense to use virus-specific primers for \ac{PCR} or \ac{NGS} 

``Comparison of the whole genome sequences of recent \ac{LSDV} isolates from the 2015–2016 epidemic in southern Europe revealed only a limited number of point mutations between the isolates'' \ac{WGS} is essential to capture all genetic variation at once

In sequencers, false positive variants (\ac{FPV}) must be avoided (happens when too many amplification cycles are made)

BWA-MEM vs. BWA-MEM2 for small viral genomes: no big speed gain, error-prone as a new tool. preferably stick to known tools in Galaxy that are proved to work well and have a maintenance

-> caveat of tools with small development teams: abundance of technological advancements or error solving, but dependence on the reliability -> choose established software

% Efficiency: Assembly vs. Mapping; efficiency
% Kraken2 vs. VAPOR; 
% Efficiency: LoFreq vs. iVar consensus; both consensus identification methods using the same site-specific depth threshold

% building index is expensive (BWT)

FMDV: This user-involving method ensures that the consensus sequence for each sample is generated based on the closest available reference sequence, which can improve the accuracy and reliability of the final assembly. 

\section{Limitations}
\begin{itemize}
    \item only Illumina seq. data
    \item VAPOR 
    \item amplicon scheme for poxviruses
    \item mapping-based approach reaches its limits for FMDV -> requires user interaction and choice of ref. 
    \item AIV, poxvirus and FMDV not tested with many samples
\end{itemize}


\section{Future Directions}
\begin{itemize}
    \item validate with more samples/tests
    \item extend to more NGS datatypes
    \item replace tools with faster/newer tools (if available, stable and reliable on Galaxy)
    \item apply to more viruses
    \item enhance surveillance, write more training material, use pipelines in production
\end{itemize}

further validation and improvement of the developed pipelines, application to other viral livestock diseases, integration with existing surveillance systems; expand the \ac{VETLAB} network to entitle even more professionals to professionally analyse their samples.
more testing, more training material

workflows offer many possible directions for downstream analysis that could be integrated into the main workflows if needed:

* consensus sequence for each segment -> compare consensus sequence to others can help identify outbreaks and patterns of transmission, get more insights how the virus spreads and its evolution
* Prokka annotation file. Predict the protein coding regions of the virus, to understand the function of the viral proteins and how they interact with host cells
* SNPs relative to the reference sequence
* \ac{MSA} and phylogenetic tree for broad or detailed phylogenetic analysis and understand evolutionary relationships between the sample and other strains. could also use clusters or subtypes within the sample. make trees available so that new isolates can be immediately arranged
* more visualisation of the data
* long-term objective: build public high-resolution databases to enable researchers to detect mutation of an isolate. this is crucial for a global surveillance system to work.
* develop workflows for the same purpose that work with other NGS sequencing data (ONT, PacBio etc.)

fine-tune settings of individual tools, improve quality of tools (VAPOR)