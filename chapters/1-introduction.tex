\chapter{Introduction}\label{chap:introduction}
Sharing environments means sharing diseases -- this simple relationship expresses how pathogens spread among populations if they get in touch. The affected populations can be animalic or human. Impacts of disease outbreaks can be as severe as the whole world experienced during the pandemic of \ac{COVID-19} that originated in Wuhan, China in 2019. This highly contagious disease was caused by the \ac{SARS-CoV-2}, an infectious virus of presumed zoonotic origin~\cite{wu2020new}. With more than 757.26 million reported cases and more than 6.85 million confirmed deaths as of February 24, 2023~\todo{u}, this pandemic is a public health emergency that has caused estimated costs of 16 trillion U.S. dollars. Apart from this, it invoked an outstanding interest in virology research~\cite{covid}. \\
% update numbers \url{https://covid19.who.int/table}
Professionals from many different fields, such as public health specialists, researchers, biomedical staff, bioinformaticians and veterinarians are carefully monitoring potentially dangerous viral diseases by examining the viral genome. International managing institutions with a globally distributed network work on safe and healthy environments for both animal and human populations. The \ac{WOAH}, founded as \ac{OIE}, implements standards in animal health and the handling of zoonoses and other diseases. As an intergovernmental organisation following the multidisciplinary One Health principle, it supports its members in the prevention of animal diseases of concern. National veterinary authorities must notify the \ac{WOAH} in case they detect cases of diseases that are listed by the \ac{WOAH}. Modern and high-resolution monitoring of viral diseases include the sequencing of samples with \ac{NGS} platforms and inspecting the viral genome on a base-by-base level. In order to generate the full-length genome sequence in a quality that reliably constitutes the viral genome from the sample, bioinformatic steps are required. Since the motivation for the genomic analysis is given by the large impact and importance in surveillance,
these topics are examined below.

\section{Viral Livestock Diseases}\label{sec:viral} 
Infectious diseases caused by viruses that affect domesticated animals, like for example cattle, pigs, goats, sheep, and poultry are referred to as viral livestock diseases. The most frequent and known diseases include Foot-and-Mouth Disease, African swine fever, avian influenza and Newcastle disease. They can spread quickly among animals, and in some cases are transmitted from an animal host to humans, making them zoonotic diseases. There are over 200 known types of zoonoses, some of them, like rabies, being 100\% preventable through vaccination and medication~\cite{who2020zoon}. %A report from the \ac{ILRI} states that zoonoses account for approximately 2.5 billion illness cases in humans and 2.7 million deaths annually~\cite{grace2012mapping}. The \ac{CDC} and its U.S. government partners listed the top eight zoonotic diseases of national concern in a report, filing zoonotic influenza and emerging coronaviruses such as \ac{SARS} and \ac{MERS}~\cite{brown2006recent}. This joint report is used to tackle the listed diseases with a broader focus~\cite{centers8zoonotic}. At the same time, not all livestock diseases of viral origin are zoonotic: Around 60\% of all known human infectious diseases and approximately 75\% of all newly emerging infections are zoonotic~\cite{jones2008global}.

The term livestock is vague, and generally refers to any breed or animal population that is kept by humans for commercial or useful purpose. According to the 20th Livestock Census of the Department of Animal Husbandry and Dairying, given out by the Indian government, India holds the world's largest amount of livestock with 535.78 million animals as of 2019~\cite{livestock2019}. %Globally, the ice-free surface that is dedicated to the purpose of livestock whether it is for farmlands or feed production, is up to 26\% of the area~\cite{steinfeld2006livestock}. 
Not only food production and economy, but also global trade, the agricultural sector and employment rates highly depend on livestock resources. These numbers illustrate the impressive interconnectedness between humans and livestock. The consequences of a collapse of the livestock industry would therefore be significant and far-reaching. %As farming animals are directly affected by the occurrence of zoonoses in both developed and developing countries, affected parties have a strong interest in avoiding any constraints that might be caused by disease outbreaks.

\subsubsection*{Historic Outbreaks of Zoonotic Diseases}
Historically, zoonoses have shaped serious infectious events. Pathogens that cause zoonotic diseases are viruses (37.7\%), and according to surveillance data also bacteria (41.4\%), parasites (18.3\%), fungi (2.0\%) or prions (0.8\%)~\cite{salyer2017prioritizing}. Prior to the \ac{COVID-19} pandemic modern zoonotic diseases like Ebola virus disease and salmonellosis had high infection rates. Influenza viruses cause epidemics each year, and circulate in all parts of the world. Influenza appears in zoonotic and human-only spreads, but the different types of virus can recombine occasionally and cause events such as the 1918 Spanish flu~\cite{garten2009antigenic, gibbs2001recombination}. %Especially for poultry, \ac{HPAI} of the H5 subtype is an ongoing threat~\cite{lee2017evolution}.
Since the first detection of \ac{HPAI} of the H5 subtype in China, 1996 it has been reported in many avian populations worldwide, both domestic and wild. Even though it has adapted to birds as the specific host, the virus can further adapt, spillover to humans and in rare cases be transmitted between humans~\cite{webster1992evolution}. Avian influenza has caused recent seasonal outbreaks, such as the 2014-15 outbreak in the United States resulting in almost 50 million birds that died as a consequence of an infection or of depopulation~\cite{lee2016highly}. This is roughly a third of the national stock of laying hens. In 2020, there were several outbreaks reported in Europe, almost all with \ac{HPAI} viruses from the H5 subtype~\cite{lewis2021emergence}. It mainly affected farmed ducks due to the high density of animals in the facilities and the separation from wild birds due to domestication~\cite{lewis2021emergence}. The latest outbreak of \ac{HPAI} is spreading worldwide. Having started in early 2022, until today, February 23, 2023 ~\todo{u} it has led to more than 58 million culled or died birds. Different H5 subtypes have been reported in 37 countries and so far, six human infections were reported in this outbreak~\cite{authority2023avian}. This number is not nearly as high as for the animals affected, but considering that from 2003 to 2022, there were a total of 868 confirmed cases of H5N1 in humans with a mortality rate of 52\%, each human infection is a risk~\cite{authority2023avian}.
% update numbers \url{https://efsa.onlinelibrary.wiley.com/doi/abs/10.2903/j.efsa.2023.7786}

\subsubsection*{Risk Factors and Impact of Disease Outbreaks}
Reasons for recurring huge outbreaks of viral diseases in animal confinements are the advantageous circumstances for virus transmission, since it is warm and humid. In general, animal husbandry practises have evolved in the sense that domestic animal species are raised in relatively small and usually confined spaces at a high density. This domestication has given plenty of opportunities to develop more pathogens of viral and bacterial origin over time. The spread of international trading of farm animals has amplified the number of infected animals and the number of infectious diseases. \\
As transmission routes can differ depending on the disease, another factor is transmissibility, determining how easy the infectious agents spread. Vector-borne diseases are transmitted by living organisms that transfer pathogenic microorganisms to other, uninfected animals or humans. Vectors can be mosquitoes, fleas or ticks. %Among others, the \ac{WHO} identifies major globally present vector-borne diseases as malaria, dengue, yellow fever and Zika virus disease~\cite{world2017global}. 
Another transmission mode is direct contact airborne transmission. Environmental factors such as a high temperature, humidity and precipitation can facilitate a virus to spread and keep it alive~\cite{eccles2002explanation}. Inadequate food and water supplies, overpopulation and mass migration of animals pose additional risks for transmission of animal diseases in farming surroundings. \\
Outbreaks of livestock diseases do not only affect animal and human health, but also cause high economic losses. Restrictions and containment measures, as well as the culling of animals lead to loss of income for farmers -- since livestock and their products, such as milk, eggs or meat, are used for further production, other businesses that rely on these products are also affected by disease outbreaks. %Even if infected animals do not die or have to be culled, the long-term consequences of infection can affect the health of the animals.
This can reflect in poor growth, production and feed conversion. Another impact of depopulating infected animal populations is the loss of biodiversity within endangered species of wildlife populations~\cite{lacroix2014non, morand2020emerging, reid2010global}.% Wildlife populations of endangered species experiencing disease outbreak can be decimated, leading to ecological imbalances and interference with natural food chains~\cite{reid2010global, civitello2015biodiversity, espinosa2020infectious}. As shown, the spread of viral diseases among animal populations can have enormous impacts on dependent industries, individuals and populations.

\subsubsection*{Notifiable Animal Diseases}
For biosecurity and surveillance purposes, the \ac{WOAH} has agreed on a list of notifiable animal diseases that must be reported to in agricultural authorities. This list includes a total of 117 diseases, partly endemic or highly transmissible, such as Foot-and-mouth disease, lumpy skin disease, peste des petits ruminants, classical swine fever, highly pathogenic avian influenza and Newcastle disease~\cite{woah2023list}. \\ % The list does not cover all known zoonoses and animal diseases since not all of them pose an actual risk. \\
Reports of illness cases of animals filed by national veterinary authorities are used to detect unusual incidents, including mortality or sickness of animals and have adverse effects on socio-economic or public health. The notifiable animal diseases include more than 50 wildlife diseases which can impact livestock health~\cite{woah2023list}. As the surveillance of viral animal diseases is still of highest priority in order to avoid expensive and dangerous outbreaks, this topic is discussed in more detail in the following introductory chapter.

\section{Prevention, Surveillance and Control}
Given the potential danger of disease outbreaks to animal and public health, the question is how to detect, monitor, control and prevent outbreaks in farm animal populations. \\
To avoid the impact that a disease outbreak can have, the best method is to avoid the disease in the first place. This leads to the principle of prevention, which has its main task in reducing the overall risk of a virus spreading. Corresponding measures are vaccinations and hygiene standards. For viral material that reassorts over time as the number of infections increases, the potential for a virus to exploit host cell genes that favour viral growth and survival may be high~\cite{fenner2017maclachlan}. %Other disease prevention practises include disinfection and good animal husbandry. Practitioners in the field or in veterinary clinics are obliged to follow this principle of prevention.
In-depth strategies to prevent viral diseases depend heavily on the characteristics of the virus, taking into account transmission mode, environmental stability, zoonotic risk and pathogenesis. Exclusion of infected livestock and vaccination of potentially infected flocks is increasingly practised worldwide~\cite{fenner2017maclachlan}. \\ %The spatial spread of viral diseases can be contained through quarantine, separation from wildlife populations, testing and regular inspections of imported animals. \\
Surveillance of viral diseases involves the collection of basic information about the disease, including incidence, prevalence and transmission patterns; the systematic and regular collection and analysis of these data is crucial to obtain a detailed overview of the spread. To gain valuable insights into the origin and characteristics of a viral genome, samples from infected animals are sequenced with modern \ac{NGS} methods to derive information from the reconstructed whole-length viral genome. The need for data has led the \ac{WOAH} to publish the above-mentioned list of notifiable diseases. Based on the data collected, authorities can inform their decisions on the allocation of resources for disease control and other containment activities~\cite{fenner2017maclachlan, who2017one}.
%Common methods for animal disease surveillance include notifiable disease reporting, laboratory-based surveillance and population-based surveillance. General awareness among veterinary diagnosticians and practitioners is another key to an effective surveillance system. Most countries have their own national veterinary authorities, coordinated by the \ac{WOAH} to enable a coordinated exchange of information~\cite{who2017one}.
Since efforts in tackling viral disease outbreaks for example by mass vaccination are very expensive, official budgets from the governments are required. This makes it a political responsibility to prevent and control animal diseases. \\
% NGS
One important component of modern and accurate surveillance systems of viral diseases is the access to relevant data. Technologies to produce \ac{DNA} sequencing data have developed to be very cost and time efficient which makes the study of infectious diseases better and faster. At the same time, the amount of \ac{DNA} sequencing data produced with \ac{NGS}, also known as \ac{HTS} platforms, prove this change. \ac{NGS} platforms include IonTorrent, Illumina and \ac{ONT}. Advances in the biotechnological application and evaluation of these data are revolutionising the field on the molecular level~\cite{suminda2022high}. Sequencing technologies take a key role in describing viral diversity in humans and animals, in detecting pathogens and co-infections, in epidemiologic research about the evolution of viral material and in metagenomic characterisation of new microbial material. This is done by constructing the parts or the complete genetic information of a virus, the genome, where the nucleic acids store this information in single or double strands in a linear or circular sequence. Algorithms and analyses of the viral genome of concern are extensively studied and developed for many different purposes. With \ac{NGS} methods, the genome sequence can be precisely determined. More detailed methods that are used for viral animal disease surveillance with \ac{NGS}-based technologies are described in~\chapref{chap:state-art}.

\section{Motivation and Objectives of the Thesis}
Bioinformatics and data analysis are crucial for understanding and monitoring viral diseases. However, there is a lack of knowledge and resources in many parts of the world. This is particularly true for poorer countries with small laboratories and national health organisations that are not well equipped with modern sequencers and surveillance systems. Additionally, transporting clinical samples across international borders is difficult and expensive. Nonetheless, efforts are made to establish global networks such as the \ac{ZODIAC}. It is an initiative by the \ac{IAEA}, launched in 2021, with five major objectives: (1) Strengthening member states' detection, diagnostic and monitoring capabilities, (2) Developing and making novel technologies available for the detection and monitoring of zoonotic diseases, (3) Making real-time decision-making support tools available for timely interventions, (4) Understanding the impact of zoonotic diseases on human health and (5) Providing access to an agency coordinated response for zoonotic diseases~\cite{zodiac2021}. In collaboration with technical experts from different fields from all over the world, and to support the \ac{VETLAB} Network, the \ac{ZODIAC} project has the resources to provide standardised, easy-access, public and integrated pipelines for virus surveillance on a long-term. This will enable laboratories and veterinarians to monitor and analyse their samples more effectively, leading to early detection and prevention of viral livestock diseases. \\
Due to the outstanding research efforts brought about by the \ac{COVID-19} pandemic, analysis pipelines for \ac{SARS-CoV-2} samples were developed on the Galaxy platform. Galaxy and the implementation of pipelines are discussed in more detail in~\chapref{chap:methods}. Reusing parts of the globally used \ac{SARS-CoV-2} pipeline with \ac{NGS} input data for genomic analysis can help to understand other viruses and ultimately lead to a deeper understanding of viral genomes from isolates. \\
This work is part of the \ac{ZODIAC} project and supports pillar (2) in the development of integrated pipelines that enable laboratories, veterinarians and other health professionals to analyse their data from samples obtained with \ac{HTS} technologies. The developed pipelines concern avian influenza A for subtype identification and genome analysis, a poxvirus pipeline for determining poxvirus genomes sequenced as half-genomes in a tiled-amplicon approach and \ac{FMD} for serotyping and genome analysis. %The poxvirus pipeline has been tested with samples of lumpy skin disease virus.
These viruses are chosen for the availability of test samples that were used for validation of the pipelines, for relevance within the \ac{ZODIAC} project and for their importance concerning animal and public health risk. All three pipelines follow an approach that relies on raw read data and enables monitoring of intra-sample minor allelic variant frequencies. The high resolution allows early warning of epidemiological signs of a changing viruses, specifically important for the assessment of emerging variants in pathogenicity and vaccine sensitivity. The aim is to provide fast, sensitive pipelines that are ready-to-use for surveillance purposes and rely on concepts from \ac{SARS-CoV-2} research. \\
In summary, the lack of bioinformatics resources in many countries poses a major challenge to effective viral livestock disease surveillance. However, established global networks such as \ac{ZODIAC} together with \ac{VETLAB} can provide the necessary resources to enable effective surveillance and analysis of viral animal diseases. This in turn will lead to early detection, insights into transmission routes and changes of the virus, prevention of disease outbreaks and ultimately protect public health and reduce the impact of viral diseases on livestock.
