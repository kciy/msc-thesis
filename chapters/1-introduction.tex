\chapter{Introduction}\label{chap:introduction}

Sharing environments means sharing diseases -- this simple relationship expresses how pathogens found in animal populations can spread to humans and have severe impacts. The impact can be as severe as the whole world experienced during the pandemic of Coronavirus Disease 19 (COVID-19) that originated in Wuhan, China in 2019. This highly contagious disease was caused by the Severe Acute Respiratory Syndrome Coronavirus 2 (SARS-CoV-2), an infectious virus of presumed zoonotic origin~\cite{wu2020new}. With more than 757.26 million reported cases and more than 6.85 million confirmed deaths as of February 24, 2023~\todo{update numbers}, this pandemic is a public health emergency that has caused estimated costs of 16 trillion U.S. dollars. Apart from this, it invoked an outstanding interest in virology research~\cite{covid}. \\
Since then, professionals from many different fields, i.e. public health specialists, researchers, biomedical staff, bioinformaticians and veterinarians have put even more effort than before into the monitoring of potentially dangerous viral diseases. International managing institutions with a globally distributed netwok work on safe and healthy environments for animal and human populations. The World Organisation for Animal Health (WOAH), founded as Office International des Epizooties (OIE), implements standards in animal health and the handling of zoonoses and other diseases. As an intergovernmental organisation following the multidisciplinary One Health principle, it supports its members in the prevention of animal diseases of concern. National veterinary authorities must notify the WOAH in case they detect cases of diseases that are listed by the WOAH. The most important definitions, the significance, impacts and surveillance measures of animal diseases are examined below.

\section{Viral Livestock Diseases}\label{sec:viral}

Infectious diseases caused by viruses that affect domesticated animals, like for example cattle, pigs, goats, sheep, and poultry are referred to as viral livestock diseases. The most frequent and known diseases include Foot-and-Mouth Disease, African Swine Fever, Avian Influenza and Newcastle Disease. They can spread quickly among animals, and in some cases are transmitted to humans, making them zoonotic diseases. There are over 200 known types of zoonoses, some of them like rabies being 100\% preventable through vaccination and medication~\cite{who2020zoon}. When viral livestock diseases become zoonotic, they pose a significant public health risk, possibly leading to widespread illness and death. A report from the International Livestock Research Institute (ILRI) states that zoonoses account for approximately 2.5 billion illness cases in humans and 2.7 million deaths annually~\cite{grace2012mapping}. The Centres for Disease Control and Prevention (CDC) and its U.S. government partners listed the top eight zoonotic diseases of national concern in a report, filing zoonotic influenza and emerging coronaviruses such as SARS and Middle East Respiratory Syndrome (MERS)~\cite{brown2006recent}. This collaborative report is used for focussing on the listed diseases since they are of greatest concern~\cite{centers8zoonotic}. At the same time, not all livestock diseases of viral origin are zoonotic: Around 60\% of all known human infectious diseases and approximately 75\% of all newly emerging infections are zoonotic~\cite{jones2008global}.

The term livestock is a vague term that generally refers to any breed or animal population that is kept by humans for commercial or useful purpose. According to the 20th Livestock Cencus of the Department of Animal Husbandry and Dairying, given out by the Indian government, India holds the world's largest amount of livestock with 535.78 million animals as of 2019~\cite{livestock2019}. Globally, the ice-free surface that is dedicated to the purpose of livestock whether it is for farmlands or feed production, is up to 26\% of the area~\cite{steinfeld2006livestock}. Not only food production and economy, but also global trade, the agricultural sector and employment rates highly depend on livestock resources. These numbers illustrate the impressive interconnectedness of the humans with the livestock sector. The consequences of a collapse of this important industry would therefore be significant and far-reaching. As the livestock industry is directly affected by the occurrence of zoonoses in both developed and developing countries, affected parties have a strong interest in avoiding any constraints that might be caused by disease outbreaks. Some diseases cause high costs for the industry every year because many animals are affected or infected and have to be culled.

\subsubsection*{Historic Outbreaks of Zoonotic Diseases}
Historically, zoonoses have shaped serious infectious events. Pathogens that cause zoonotic diseases are viruses (37.7\%), and according to surveillance data also bacteriums (41.4\%), parasites (18.3\%), fungi (2.0\%) or prions (0.8\%)~\cite{salyer2017prioritizing}. Prior to the COVID-19 pandemic modern zoonotic diseases like Ebola virus disease and salmonellosis had high infection rates. Influenza viruses cause epidemics each year, and circulate in all parts of the world. There are four types of seasonal influenza viruses (A, B, C and D), however only influenza A and B cause yearly epidemics. Influenza strains appear in zoonotic and human-only spreads, but the viruses can recombine occasionally and cause events such as the 1918 Spanish flu~\cite{garten2009antigenic, gibbs2001recombination}. Especially for poultry, highly pathogenic avian influenza (HPAI) of the H5 subtype is an ongoing threat~\cite{lee2017evolution}. Since its first case in China, 1996 it has been detected in many avian populations, both domestic and wild. The H5 subtype is the avian influenza type with the greatest risk. Even though it has adapted to birds as the specific host, the virus can further adapt and be transmitted between humans~\cite{webster1992evolution}. Avian influenza has caused seasonal outbreaks, such as the 2014-15 outbreak in the United States resulting in almost 50 million birds that died as a consequence of an infection or of depopulation~\cite{lee2016highly}. In 2020, there were several outbreaks reported in Europe, almost all with HPAI viruses from the H5 subtype~\cite{lewis2021emergence}. It mainly affected farmed ducks due to the high density of animals in the facilities and the separation from wild birds due to domestication~\cite{lewis2021emergence}. The latest outbreak of avian influenza is still ongoing, started in early 2022 and until today, February 23, 2023 ~\todo{update numbers and source} has led to more than 58 million culled or died birds. It is reported in 37 countries and so far, six human infections were reported in this outbreak~\cite{authority2023avian}. This number is not nearly as high as for the animals affected, but considering that during the last 20 years, there were fewer than 900 confirmed cases of H5N1 in humans and the mortality rate of 50\%, each human infection is a risk~\cite{authority2023avian}.

\subsubsection*{Risk Factors and Impact of Disease Outbreaks}
Reasons for recurring huge outbreaks of viral diseases in animal confinements come from the advantageous circumstances for virus transmission as it is warm and humid. In general, animal husbandry practices have evolved in the sense that domestic animal species are raised in relatively small and usually confined spaces at a high density. This domestication has given plenty of opportunities to develop more pathogens of viral or bacterial origin over time. The spread of international trading of farm animals has amplified the number of infected animals and the number of infectious diseases. As transmission routes can differ depending on the disease, the other factor is how easy the infectious agent spreads (transmissibility). Vector-borne diseases are transmitted by living organisms that transfer pathogenic microorganisms to other, uninfected animals or humans. Vectors can be mosquitoes, fleas or ticks. Among others, the World Health Organization (WHO) identifies major globally present vector-borne diseases as malaria, dengue, yellow fever and Zika virus disease~\cite{world2017global}. Another transmission mode is direct contact airborne transmission. Environmental factors such as a high temperature, humidity and precipitation can facilitate a virus to spread and keeping it alive~\cite{eccles2002explanation}. Overpopulation, inadequate food and water supplies and mass migration of populations pose additional risks for transmission of animal diseases. \\
Outbreaks of livestock diseases do not only affect animal and human health, but also cause high economic losses. Restrictions and containment measures, as well as the culling of animals in the case of confirmed cases of listed diseases, lead to a loss of income for farmers -- since livestock and their products, such as milk, eggs or meat, are used for further production, other businesses that rely on these products are also affected by disease outbreaks. Even if infected animals do not die or have to be culled, the medium- and long-term consequences of infection can affect the health of the animals. This can lead to poor growth or poor production and feed conversion. Another impact of depopulating infected animal populations is the loss of biodiversity~\cite{lacroix2014non, morand2020emerging}. Wildlife populations of endangered species experiencing disease outbreak can be decimated, leading to ecological imbalances and interference with natural food chains~\cite{reid2010global, civitello2015biodiversity, espinosa2020infectious}. \\
As shown, the spread of viral disesaes among animal populations can have enormous impacts on dependent industries, individuals and populations.

\subsubsection*{Notifiable Animal Diseases}
For reasons of biosecurity and surveillance purposes, the WOAH has agreed on a list of notifiable animal diseases that must be reported to in agricultural authorities. This list includes a total of 117 diseases, partly endemic or highly transmissable, such as Foot-and-mouth-disease, lumpy skin disease, peste des petits ruminants, classical swine fever, highly pathogenic avian influenza and Newcastle disease. The list does not cover all known zoonoses and animal diseases since not all of them pose an actual risk for costly outbreaks. \\
Reports of illness cases of animals filed by national veterinary authorities are used to detect unusual incidents, including mortality or sickness of animals and have adverse effects on socio-economic or public health. The notifiable animal diseases include more than 50 wildlife diseases, which may have impact on livestock health~\cite{woah2023list}. As the surveillance of viral animal diseases is still of highest priority in order to avoid expensive and dangerous outbreaks, this topic is discussed in more detail in the following introductory chapter.


\section{Prevention, Surveillance and Control}
Given the potential danger of disease outbreaks to animal, human and public health, the question is how to detect, monitor, control and prevent outbreaks in farm animal populations. \\
To avoid the impact that a disease outbreak can have, the best method is to avoid the disease in the first place. This leads to the principle of prevention, which sees its main task as reducing the overall risk of a virus spreading. Corresponding measures can be vaccinations and the establishment of hygiene standards. For viral material that recombines over time as the number of infections increases, the potential for the virus to exploit host cell genes that favour viral growth and survival may be high~\cite{fenner2017maclachlan}. Therefore, it seems logical to reduce the overall number of infections. Other disease prevention practises primarily include disinfection and good animal husbandry. Practitioners in the field or in veterinary clinics are obliged to follow this principle of prevention. In-depth strategies to prevent viral diseases depend heavily on the characteristics of the virus, taking into account transmission modes, environmental stability, zoonotic risk and pathogenesis. Exclusion of livestock and the use of vaccines from potentially infected flocks is increasingly practised~\cite{fenner2017maclachlan}. The spatial spread of disease can be contained through quarantine, separation from wildlife populations, testing and regular inspections of imported animals. \\
In the event of an actual outbreak of a viral animal disease, control and surveillance are key. Surveillance of viral diseases involves the collection of basic information about the disease, including incidence, prevalence and transmission patterns; the systematic and regular collection and analysis of these data is crucial to obtain a detailed overview of the spread. This need for data has led the WOAH to publish the above-mentioned list of notifiable diseases. Based on the data collected, authorities can inform their decisions on the allocation of resources for disease control and other containment activities~\cite{fenner2017maclachlan, who2017one}. \\
Common methods for animal diseases surveillance include notifiable diseases reporting, laboratory-based surveillance and population-based surveillance. General awareness among veterinary diagnosticians and practitioners is another key to an effective surveillance system. Most countries have their own national veterinary authorities, coordinated by the WOAH to enable a coordinated exchange of information~\cite{who2017one}. \\
It is vital to analyse collected data promptly in order to influence necessary follow-up actions. National databases may contain reliable and annotated data, but they often reflect information gathered several weeks or months ago. On the other hand, early warning signs of a potential disease outbreak may be found in local media reviews, unusual social media activity or unverified individual reports on the internet. However, such sources can provide well-intentioned but inaccurate information. Timely action and communication of information, particularly to local veterinaries is a crucial component of effective surveillance systems. Nevertheless it is important to exercise caution to prevent unnecessary public concern.

% NGS
One important component of modern and accurate surveillance systems of viral diseases is the access to relevant data. Technologies to produce DNA sequencing data have developed to be very cost and time efficient which makes the study of infectious diseases better and faster. At the same time, the amount of DNA sequencing data produced with next-generation sequencing (NGS; also known as high-throughput sequencing, HTS) platforms prove this change. NGS platforms include IonTorrent, Illumina HiSeq/MiSeq (for different read lengths) and Oxford Nanopore Technologies (ONT). Advances in the biotechnological application and evaluation of these data are revolutionalising the field of studying these data on the molecular level~\cite{suminda2022high}. Sequencing technologies take a key role nowadays in describing viral diversity in humans and animals, in detecting pathogens and co-infections, in epidemiologic research about the evolution of viral material and in metagenomic characterization of new microbial material. More detailed methods that are used for viral animal disease surveillance with NGS-based technologies are described in Chapter~\ref{chap:state-art}.

\section{Motivation and Objectives of the Thesis}
Bioinformatics and data analysis are crucial for understanding and monitoring viral diseases. However, there is a lack of knowledge and resources in many parts of the world. This is particularly true for poorer countries with small laboratories and national health organisations that are not well equipped with modern sequencers. Additionally, transporting clinical samples across international borders is difficult. Nonetheless, efforts are made to establish global networks such as the Zoonotic Disease Integrated Action (ZODIAC). It is an initiative by the International Atomic Energy Agency (IAEA), launched in 2021, with five major objectives: (1) Strenghthening member states' detection, diagnostic and monitoring capabilities, (2) Developing and making novel technologies available for the detection and monitoring of zoonotic diseases, (3) Making real-time decision-making support tools available for timely interventions, (4) Understanding the impact of zoonotic diseases on human health and (5) Providing access to an agency coordinated response for zoonotic diseases~\cite{zodiac2021}. In collaboration with technical experts from different fields and from all over the world, and to support the Veterinary Diagnostic Laboratory (VETLAB) Network, the ZODIAC project has the resources to provide standardised, easy-access, public and integrated pipelines for virus surveillance on a long-term. This will enable laboratories and veterinarians to monitor and analyse their samples more effectively, leading to early detection and prevention of viral livestock diseases.\\
Due to the outstanding research efforts brought about by the COVID-19 pandemic, analysis pipelines for SARS-CoV-2 samples were developed on the Galaxy platform. Galaxy and the implementation of pipelines is discussed in more detail in Chapter~\ref{chap:methods}. Using the knowledge and application of SARS-CoV-2 and transferring it to other viruses will lead to a more comprehensive understanding of viral diseases and better prevention strategies.\\
This work is part of the ZODIAC project and supports pillar (2) in the development of integrated pipelines that enable laboratories, veterinarians and other health professionals to analyse their data from samples obtained with HTS technologies. The zoonoses studied are avian influenza A for subtype identification and a poxvirus pipeline for determining poxvirus genomes sequenced as half-genomes in a tiled-amplicon approach. This pipeline has been tested with samples of lumpy skin disease virus. These two viruses have been chosen because of available test samples that were used for validation of the pipelines. \\
In summary, the lack of bioinformatics knowledge and resources in poorer countries poses a major challenge to effective, globally integrated viral animal diseases surveillance systems. However, established global networks such as ZODIAC together with VETLAB can provide the necessary resources to enable effective surveillance and analysis of viral animal diseases. This in turn will lead to early detection and prevention of disease outbreaks and ultimately protect public health and reduce the impact of viral diseases on livestock.
