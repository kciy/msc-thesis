\chapter{Conclusion}\label{chap:conclusion}
Whole-genome sequencing data provide valuable information on viral samples of emerging livestock diseases. Automated pipelines, accessible on the community-based and open-sourced Galaxy platform, solve the complex task of reconstructing the consensus sequence from raw read data. The developed workflows carefully take into account virus-specific characteristics for high-quality genomic analysis, while relying on the concept of reference-based mapping instead of error-prone and computationally expensive \textit{de novo} assembly. The decision on a representative reference is integrated into each workflow by database searches. Especially for \ac{AIV}, reassortment events occurring in the individual gene segments challenge the choice of appropriate reference sequence, what makes it difficult to agree on a single sample that fits in all eight segments. Relying directly on raw read data rather than assembled contigs and allowing the backtracing to the raw data, integrating the dynamic reference selection based on a classification search in a large database goes beyond current influenza surveillance pipelines.\\
Poxviruses are attributed with identical repeated regions within the genome where unambiguous alignment regularly fails and in order to solve this challenge, a tiling amplicon approach is used. It requires sequencing of the sample in two pools and executes mapping in two steps to separate the ambiguous positions.\\
For shorter viral genomes such as \ac{FMDV}, we showed that \textit{de novo} assembly is reasonably faster than for large \ac{DNA} genomes and serves as a starting point for a reference search for mapping. It involves the user deciding on a reference sequence by reviewing the megablast search results. In this way, possibly unexpected contamination or co-infection in the sample can be detected. While all three approaches are highly specific for the viral attributes, they share common goals with in terms of lineage identification and allelic variant extraction, and rely on approved concepts used in other existing pipelines like such for genomic sequence analysis of \ac{SARS-CoV-2}.\\
Combining steps of proper preprocessing and filtering to reduce misalignments and the incorporation of sequencing errors into the alignment, downstream analysis based on alignment files and consensus sequence like variant calling, phylogenetic embedding and gene annotation are missing concepts in the workflows. Still, the workflows serve as starting points for more, specialised surveillance efforts.
