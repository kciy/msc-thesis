\chapter{Discussion}\label{chap:discussion}

\section{Contribution to the Field}
single sample vs. multi sample (reality check, what is needed?) \\
further pox viruses, pipelines can be more or less easily applied/adjusted \\
limitations \\
LSDV für alle Pox-Viren interessant \\
AIV downstream alles. Es wäre gut key minor assets zu highlighten, die auf Adaption bei Säugetiere hinweist -> Databases werden benötigt zum Abgleichen ob ein Isolat mutiert ist? \\
Generell will man auf Aminosäure-Ebene annotieren (meiste Information) \\

Stammbäume zugänglich öffentlich, wäre gut die öffentlich zu haben, auch detailliert also >1 Sample pro Land, sehr feingliedrig um echt einordnen zu können (1 Isolat pro Kontinent bringt nicht so viel)

\section{Future Directions}
further validation and improvement of the developed pipelines, expansion to other viral livestock diseases, integration with existing surveillance systems; expand the VETLAB network to entitle even more professionals to professionally analyse their samples.

AIV workflow offers many possible directions for downstream analysis:

* consensus sequence for each segment -> compare consensus sequence to others can help identify outbreaks and patterns of transmission, get more insights how the virus spreads and its evolution
* Prokka annotation file. Predict the protein coding regions of the virus, to understand the function of the viral proteins and how they interact with host cells
* SNPs relative to the reference sequence
* MSA and phylogenetic tree for broad or detailed phylolgenetic analysis and understand evolutionary relationships between the sample and other strains. could also use clusters or subtypes within the sample. make trees available so that new isolates can be immediately arranged
* more visualisation of the data

* long-term objective: build public high-resolution databases to enable researchers to detect mutation of an isolate. this is crucial for a global surveillance system to work.